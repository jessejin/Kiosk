\documentclass[12pt,a4paper,ngerman]{scrartcl}

\usepackage{babel}
\usepackage{graphicx}
\usepackage[T1]{fontenc}
\usepackage{lmodern}
\usepackage{wrapfig}
\usepackage{fancyhdr}
\usepackage{lastpage}
\usepackage{amsmath}

% Unicode math fonts
\usepackage{unicode-math}
\setmathfont{Asana-Math.otf}
% \setmathfont{xits-math.otf}
% \setmathfont{STIXGeneral}
\newcommand{\lojoin}{⟕}
\newcommand{\rojoin}{⟖}
\newcommand{\fojoin}{⟗}


\pagestyle{fancy}
\cfoot{\thepage\ von \pageref{LastPage}}
\title{KioskLauncher: \\ Ein Kioskmodus für Android}
\subtitle{Android Praktikum \\ Sommersemester 2011}
\begin{document}
\maketitle
\section*{Das Team}
\begin{tabular}{ l l l }
  \textbf{Christopher Loessel} & \textbf{Steffen Arntz} & \textbf{Daniel} \smallskip \\ 
  JID: cloessl@gmail.com & JID: winrootkit@googlemail.com & ICQ/TEL \\
  Leitung & Konzepte/Documentation & Backend/Verification/Testing \\
\end{tabular}

\section*{Die Idee}
Smartphones und Tablet sind eine außergewöhnliche Möglichkeit multimediale Inhalte an Benutzer zu vermuitteln, jedoch ist ihr Funktionsumfang und die Komplexität für viele Anwendungen weit mehr als Ausreichend. 
Wird ein Gerät nun für einen spezifischen Einsatzzweck/Ort genutzt, so kann dies hinderlich sein, da der Nutzer ggf. zu viel Einarbeitungszeit benötigt oder sogar Fehler in der Bedienung begehen kann. \\
Als Lösung bietet es sich an den Zugriff gezielt auf bestimmte Funktionen und Anwendungen des Gerätes ein zu schränken. \\
Ein solches Gerät ließe sich mit sehr kurzer Einführung, ggf. duch ein einfaches Video, zum Beispiel auch von Besuchern einer Austellung oder eines Museeums nutzen. Außerdem ist die Nutzbarkeit auf eine Spezifisches Gebiet eingeschränkt wodurch der Reiz zum Diebstahl geschwächt wird. \\
Wir schlagen die Entwicklung einer Applikation vor, die dem Benutzer eine vereinfachte und intuitive Oberfläche bietet und ihm einfachen Zugriff auf ausgewählte Anwendungen und Informationen ermöglicht, zudem soll er Zufgriff auf eine kurze Einführung in die Bedienung erhalten.

\end{document}
