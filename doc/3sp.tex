\documentclass[12pt,a4paper,ngerman]{scrartcl}

\usepackage[ngerman]{babel}
\usepackage[babel]{csquotes}
\usepackage{graphicx}
\usepackage[T1]{fontenc}
\usepackage{lmodern}
\usepackage{wrapfig}
\usepackage{fancyhdr}
\usepackage{lastpage}
\usepackage{amsmath}

% Unicode math fonts
\usepackage{unicode-math}
\setmathfont{Asana-Math.otf}
% \setmathfont{xits-math.otf}
% \setmathfont{STIXGeneral}
\newcommand{\lojoin}{⟕}
\newcommand{\rojoin}{⟖}
\newcommand{\fojoin}{⟗}

\usepackage{xspace}
\newcommand{\zB}{\mbox{z.\,B.}\xspace}


\pagestyle{fancy}
\cfoot{\thepage\ von \pageref{LastPage}}
\title{KioskLauncher: \\ Ein Kioskmodus für Android}
\subtitle{Android Praktikum \\ Sommersemester 2011}
\begin{document}
\maketitle 

\section*{Das Team}
\begin{tabular}{ l l l }
  \textbf{Christopher Loessel} & \textbf{Steffen Arntz} & \textbf{Daniel Fischer} \smallskip \\ 
  JID: hashier@x-berg.de & JID: winrootkit@googlemail.com & ICQ: 162978399 \\
  eMail: c.loessl@tu-bs.de & eMail: winrootkit@googlemail.com & eMail: d.fischer@tu-bs.de  \\
  Controller & Concept/Documentation & Verification/Testing \\
\end{tabular}

\section*{Die Idee}
Smartphones und Tablets sind eine außergewöhnliche Möglichkeit multimediale Inhalte an Benutzer zu vermitteln, jedoch ist ihr Funktionsumfang und die Komplexität für viele Anwendungen weit mehr als Ausreichend. 
Wird ein Gerät nun für einen spezifischen Einsatzzweck/Ort genutzt, so kann dies hinderlich sein, da der Nutzer ggf. zu viel Einarbeitungszeit benötigt oder sogar Fehler in der Bedienung begehen kann. \\
Als Lösung bietet es sich an den Zugriff gezielt auf bestimmte Funktionen und Anwendungen des Gerätes einzuschränken. \\
Ein solches Gerät ließe sich mit sehr kurzer Einführung, ggf. durch ein einfaches Video, zum Beispiel auch von Besuchern einer Ausstellung oder eines Museums nutzen. Außerdem ist die Nutzbarkeit auf eine Spezifisches Gebiet eingeschränkt wodurch der Reiz zum Diebstahl geschwächt wird. \\
Wir schlagen die Entwicklung einer Applikation vor, die dem Benutzer eine vereinfachte und intuitive Oberfläche bietet und ihm einfachen Zugriff auf ausgewählte Anwendungen und Informationen ermöglicht, zudem soll er Zugriff auf eine kurze Einführung in die Bedienung erhalten.

\section*{Technische Umsetzung}
Um dem Benutzer die Möglichkeiten zu nehmen an nicht erlaubte Bedienelemente/Einstellungen zu kommen, wird eine neu Implementation des Desktopmanager (Home-Menu) angestrebt. In diesem Home-Menu ist eine eigene Implementierung des Launcher vorgesehen um die freigeschalteten Programme zu starten. \zB Google-Maps oder Wapedia (eine Anwendung um Wikipedia auf Smartphones zu durchforsten).

\section*{Anforderungen}
Der Benutzer darf nicht aus der SandBox ausbrechen können und darf nur dort definierte Anwendungen starten können. \\
Der Administrator darf natürlich aus dieser SandBox \enquote{ausbrechen} um das Android-System normal benutzen und verwalten zu können. \\
Der Administrator kann über die Einstellungen die Activities/Intents auswählen die in der SandBox benutzt/gestartet werden dürfen.
\subsection*{Optional}
Ein Defaultbrowser mit eingeschränktem Internetzugang, \zB nur eine vorgegebene Domain. \\
Default-Policy ist aus sicherheitsgründen natürlich \enquote{alles was nicht erlaubt ist, ist verboten}\footnote{Leider ist es scheinbar nicht möglich die Statuszeile mit Android 2.3 dauerhaft auszublenden}.

\section*{Milestones}
\begin{tabular}{ l l l }
Androidprojekt anlegen und Bibliotheken Recherche & 04.04.11 & 18.04.11 \\
Grundlagen und Ausführung evaluieren & 04.04.11 & 15.04.11 \\
Klassen und Programmgerüst erstellen & 04.04.11 & 28.06.11 \\
\textbf{Erstellung 3-SP} & \textbf{05.04.11} & \textbf{13.04.11} \\
GUI Prototypentwurf Graphisch & 13.04.11 & 22.04.11 \\
GUI Prototypentwurf XML & 22.04.11 & 02.05.11 \\
Erste Funktionstests (Apps starten, als root anmelden) & 18.04.11 & 15.05.11 \\
\textbf{Review 1} & \textbf{16.05.11} & \textbf{17.05.11} \\
Konfiguration implementieren & 17.05.11 & 30.05.11 \\
Einstellungsüberwachung implementieren (WLAN ein usw.) & 17.05.11 & 11.06.11 \\
App, Funktionen \& GUI verfeinern & 17.05.11 & 18.06.11 \\
Testenszenarios entwerfen und auswerten & 11.06.11 & 26.06.11 \\
\textbf{Review 2} & \textbf{27.06.11} & \textbf{28.06.11} \\
Abschlusspräsentation erstellen & 29.06.11 & 11.07.11 \\
\textbf{Projektabschluss} & \textbf{13.07.11} \\
\end{tabular}

\section*{Caveat}
Wie es bei jedem System ist an das man Hardware zugriff hat, kann man es natürlich nicht vor Manipulationen schützen die Hardwareseitig durchgeführt werden. So ist es uns nicht möglich den Angreifer vor einem Hardware übergriff \zB durch Flashen des Systems zu verhindern, geeignete Plattformauswahl kann jedoch auch hier Sicherheit bieten. Außerdem sind bestimmte UI-Gegebenheite die den Windowmanager betreffen nicht vollständig beeinflussbar.
\\ \\
\begin{center}
\begin{tabular}{@{}l@{}}\hline
Christopher Loessel
\end{tabular} \hspace{1cm}
\begin{tabular}{@{}l@{}}\hline
Steffen Arntz
\end{tabular} \hspace{1cm}
\begin{tabular}{@{}l@{}}\hline
Daniel Fischer
\end{tabular} 

\end{center}

\end{document}
